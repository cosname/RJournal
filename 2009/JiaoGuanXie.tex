\title{Conference Review: The 2nd Chinese R Conference}
\author{by Jing Jiao, Jingjing Guan and Yihui Xie}

\maketitle

The 2nd Chinese R Conference was successfully held in December 2009
in Beijing and Shanghai respectively. This conference was organized
in two cities so that more people would be able to enjoy similar
talks without additional traveling cost. The Beijing session, held
in the Renmin University of China (RUC) on December 5 -- 6, 2009,
was sponsored and organized by the Center for Applied Statistics and
the School of Statistics, RUC. Yanping Chen served as the chair of
the conference while Janning Fan was the secretary. Yixuan Qiu and
Jingjing Guan were in charge of local arrangements. The Shanghai
session took place in the East China Normal University (ECNU) on
December 12 -- 13, 2009; it was organized by the School of Resources
and Environment Science (SRES) and the School of Finance and
Statistics, ECNU, and sponsored by Mango Solutions. Jing Jiao and
Xiang Zhang served as the chairs of the conference; the Molecular
Ecology Group in SRES was in charge of local arrangements. Both
sessions were co-organized by the web community ``Capital of
Statistics'' (\url{http://cos.name}).

Beside promoting the communication among Chinese R users as the 1st
Chinese R Conference did, the 2nd Chinese R Conference concentrated
on the applications of R in statistical computing and graphics in
various disciplines. The slogan of this conference was ``useR
eveRywheRe'', indicating R is wide-spread in a variety of fields as
well as there are a large amount of useRs in China now.

In total, more than 300 participants from over 90 institutions took
part in this conference. There were 19 talks in the Beijing session:

\begin{description}
\item [{Opening}] a review on the history of the Chinese R Conference and
a brief introduction to R by Yanping Chen;
\item [{Statistical Graphics}] introduction to R base graphics by Tao Gao and Cheng
Li; matrix visualization using the package \pkg{corrplot} by Taiyun
Wei;
\item [{Statistical\ Models}] nonparametric methods and robust estimation
of quantile regression by Chen Zuo; R and WinBUGS by Peng Ding; Grey
System Theory in R by Tan Xi;
\item [{Software\ Development}] integrating R into C/C++
applications using Visual C++ by Yu Gong; using R in online data
analysis by Zhiyi Huang;
\item [{Industrial Applications}] R in establishing standards for the food industry
by Qiding Zhong; investigation and monitoring on geological
environment with R by Yongsheng Liu; R in marketing research by
Yingchun Zhu; Zhiyi Huang also gave an example on handling
atmosphere data with R; R in near-infrared spectrum analysis by Die
Sun;
\item [{Data\ Mining}] Jingjing Guan introduced \pkg{RExcel}
with applications in data mining and the trend of data mining
focused on ensemble learning; dealing with large data and reporting
automation by Sizhe Liu;
\item [{Kaleidoscope}] discussing security issues of R by Nan Xiao; using R in economics and econometrics
by Liyun Chen; R in psychology by Xiaoyan Sun and Ting Wang; R in
spatial analysis by Huaru Wang; optimization of molecular structure
parameter matrix in QSAR with the package \pkg{omd} by Bin Ma;
\end{description}

More than 150 people from nearly 75 institutions all over China
participated in the Shanghai session. Among these participants, we
were honored to meet some pioneer Chinese useRs such as Prof Yincai
Tang, the first teacher introducing R in the School of Finance and
Statistics of ECNU. There were 13 talks given in the Shanghai
session which partially overlapped with the Beijing session:

\begin{description}
\item[Opening] introduction of the 2nd Chinese R conference (Shanghai
session) by Jing Jiao, and opening address by Prof Yincai Tang and
Prof Xiaoyong Chen (Vice President of SRES);

\item[Theories] Bayesian Statistics with R and WinBUGS by Prof Yincai Tang; Grey
System Theory in R by Tan Xi;

\item[Graphics] introduction to R base graphics by Tao Gao and Cheng Li; matrix
visualization using the package \pkg{corrplot} by Taiyun Wei;

\item[Applications] using R in economics and
econometrics by Liyun Chen; marketing analytical framework by
Zhenshun Lin; decision tree with the \pkg{rpart} package by Weijie
Wang; optimization of molecular structure parameter matrix in QSAR
with the package \pkg{omd} by Bin Ma; survival analysis in R by Yi
Yu; the application of R and statistics in the semiconductor
industry by Guangqi Lin;

\item[Software Development]: dealing with large data and reporting
automation by Sizhe Liu; JAVA development and optimization in R by
Jian Li; discussing security issues of R by Nan Xiao;
\end{description}

On the second day there was a training session for R basics
(learning R in 153 minutes by Sizhe Liu) and extensions (writing R
packages by Sizhe Liu and Yihui Xie; extending R by Jian Li); a
discussion session was arranged after the training. The conference
was closed with a speech by Xiang Zhang.

In all, participants were very interested in the topics contributed
on the conference and the discussion session has reflected there was
the strong need of learning and using R in China. After this
conference, we have also decided to make future efforts on:

\begin{itemize}
  \item increasing more sessions of this conference to meet the demand of useRs in other areas;
  \item promoting the communication between statistics and other disciplines via using R;
  \item interacting with different industries through successful applications of R;
\end{itemize}


All the slides are available online at
\url{http://cos.name/2009/12/2nd-chinese-r-conference-summary/}. We
thank Prof Xizhi Wu for his consistent encouragement on this
conference, and we are especially grateful to Prof Yanyun Zhao, Dean
of the School of Statistics of RUC, for his great support on the
Beijing session. We look forward to the next R conference in China
and warmly welcome more people to attend it. The main session is
tentatively arranged in the Summer of 2010. Inquiries and
suggestions can be sent to \email{ChinaR-2010@cos.name} or
\url{http://cos.name/ChinaR/ChinaR-2010}.



\address{Jing Jiao\\
  School of Resources and Environment Science\\
  East China Normal University\\
  China P. R.}\\
\email{jing.jiao@cos.name}

\address{Jingjing Guan\\
  School of Statistics, Renmin University of China\\
  China P. R.}\\
\email{jingjing.guan@cos.name}

\address{Yihui Xie\\
  Department of Statistics \& Statistical Laboratory\\
  Iowa State University\\
  USA}\\
\email{yihui.xie@cos.name}
