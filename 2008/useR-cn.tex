\title{The 1st Chinese R Conference}
\subtitle{and R speaks Chinese}
\author{by Yihui Xie}

\maketitle

The first Chinese R conference took place at the Renmin University
of China (RUC), Beijing, China, December 13 -- 14, 2008. The
conference was organized and sponsored by the Center for Applied
Statistics, RUC, and co-organized by the School of Statistics, RUC.
Yihui Xie served as the chair of the conference and program
committee; Qingping Zhang and Haoyu Yu were in charge of local
arrangements.

Due to the lack of communication among Chinese R users in the past
as well as increasing need of using and learning \R{} for users in
China, this pioneer meeting mainly focused on

\begin{itemize}
  \item introducing and popularizing \R{} as a powerful tool for statistical
  computation and graphics;
  \item gathering Chinese \R{} users and promoting communication
  between different disciplines and industries;
\end{itemize}

Nearly 150 participants from 70 institutions all over China met in
Beijing and heard 25 talks in the two-day conference, among which we
were honored to meet several Chinese pioneer useRs such as Prof
Xizhi Wu (who was the first person introducig \R{} in the School of
Statistics, RUC more than 7 years ago) and Guohui Ding (who
contributed a lot of work in translating \R{} manuals). Besides, we
have also gained various support from overseas leading \R{}/S people
such as Richard Becker, John Maindonald (who was invited and tried
to give a remote talk at the conference via Skype!) and Kurt Hornik.

The conference program included 13 sessions on several topics:

\begin{description}
  \item[Opening] introduction of the first \R{} conference by Yihui
  Xie and opening address by Prof Xizhi Wu;
  \item[Introduction to \R{}] history and development of \R{} by Guohui Ding
  and Yihui Xie respectively, and \R{} basics by Sizhe Liu;
  \item[Statistics Basics] survey data analysis by Peng Zhan and
  statistical simulation by Tan Xi;
  \item[Biostatistics] introduction to Bioconductor and its application in bioinformatics
  by Gang Chen, application of \R{} in genetics by Liping Hou, and research on biological
  competition by Hong Yu;
  \item[Data Mining] data mining with \R{} by John Maindonald and
  Yihui Xie, introduction to various data mining methods in \R{} by
  Sizhe Liu, and \R{} in business intelligence by Jian Li;
  \item[Statistical Models] quantile regression in \R{} by Chen Zuo;
  \item[Bayesian Statistics] introduction to Bayesian statistics in
  \R{} by Peng Ding;
  \item[Statistical Computation] optimization in \R{} by Taiyun Wei,
  and simulation and inference of stochastic differential equations
  using \R{} by Yanping Chen;
  \item[Software Development] \R{} web applications by Chongliang
  Li, and integration of \R{} with Microsoft Office via \R{} (D)COM
  server by Jian Li;
  \item[R in Industries] visualization of complex system by Xiang
  Zhang, fitting and projections of mortality stochastic models
  based on \R{} and Excel by Yunmei Weng, and
  statistics and \R{} in semiconductor industry by Ming Liu;
  \item[Teaching] introduction to the \R{} package \pkg{animation}
  by Yihui Xie;
  \item[Kaleidoscope] application of \R{} on hydrological modeling
  by Huaru Wang, turning from SAS to \R{} by Jian Wang, using \R{}
  in Quantitative Structure-Activity Relationship by Bin Ma,
  exploring irregular data with \R{} by Yihui Xie;
  \item[Publication] discussion of publishing \R{}-related materials
  in China by Xiaojie Han;
\end{description}

Participants were really amazed to see the wide applications of \R{}
and made warm discussions after each presentation; session chairs
often tried hard to control the time for discussion. This has
reflected the strong need of learning and using \R{} for Chinese
users.

In the end, we decided to make future efforts on

\begin{itemize}
  \item filling the gaps between statistics and other disciplines
  with the help of \R{};
  \item localization of \R{}, e.g. form a special group to translate
  more materials on \R{};
  \item publishing more Chinese books and papers on \R{} so that
  users could get started more easily (after this conference, there
  will be proceedings for publication);
\end{itemize}

All slides and pictures can be found in this Chinese web page:
\url{http://cos.name/2008/12/1st-chinese-r-conference-summary/}.
This conference was partially supported by the website
\url{http://cos.name}, which is one of the largest statistical
websites and main places for \R{} help in China (see the board
``\href{http://cos.name/bbs/thread.php?fid=15}{S-Plus \& \R{}}'' in
the forum). We look forward to the next R conference in China and
warmly welcome more people (including overseas friends) to attend
it.

\address{Yihui Xie\\
School of Statistics, Room 1037, Mingde Main Building,\\
Renmin University of China, Beijing, 100872, China}
\email{xieyihui@gmail.com}
